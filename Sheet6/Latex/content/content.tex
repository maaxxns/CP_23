
\section{Solution}
\label{sec:auswertung}


\begin{itemize}
    \item[a)]
        In the second exercise we were asked to solve the Lorentz equations 
        \begin{align}
            \dot{X} = & -\sigma X + \sigma Y \\
 ̇           \dot{Y} = & -XZ + rX - Y \\
 ̇           \dot{Z} = & XY - bZ
        \end{align}
        with the fourth order Runge-Kutta scheme.
        The implementation can be found in file "2.cpp".
        The parameters $r, \sigma$ and $b$ were given as:
        \begin{align}
            r = & 20 \text{or} 28\\
            \sigma = & 10 \\
            b = & 8/3
        \end{align}
    \item[b)]
        After the implementation we were asked to visualize our solution.
        This should be done in three ways
        \begin{itemize}
            \item[1.]
                First as a projection of the trajectory on the xy-plane.
                \begin{figure}
                    \includegraphics[width=\textwidth]{images/Lorentz_r_20_projection.pdf}
                    \caption{The projection of the trajectory onto the xy-plane with the starting parameter $r=20$.}
                \end{figure}
                \begin{figure}
                    \includegraphics[width=\textwidth]{images/Lorentz_r_28_projection.pdf}
                    \caption{The projection of the trajectory onto the xy-plane with the starting parameter $r=28$.}
                \end{figure}
            \item[2.]
            As a Poincare slice at $Z=20$ with the condition that $\dot{Z} < 0$.
            We tested for this condition by subtracting the $\vec{f}(t_i)$ with $\vec{f}(t_{i+1})$. 